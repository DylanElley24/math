\title{Assignment 1}

\date{\today}

\documentclass[12pt]{article}

\begin{document}
\maketitle


\section*{Question 1}
\begin{displaymath}
\begin{array}{|c c|c|c|c|}

P & Q & (P \land Q) & (P \lor Q) & (P \land Q) \Rightarrow (P \lor Q)\\
\hline 
T & T & T & T & T\\
T & F & F & T & T\\
F & T & F & T & T\\
F & F & F & F & T\\
\end{array}
\end{displaymath}

\paragraph{Conclution}
This truth table demonstrates that no matter the states of $P$ and $Q$ the results will always be true for the logical expression:\\ $(P \land Q) \Rightarrow (P \lor Q)$.

\section*{Question 2}
Let $n$ be an integer.  If $3n+ 1$ is even then $n$ is odd.
\paragraph{Proof}
Assume $3n + 1$ is even, therefore we can say:\\
$3n + 1 = 2m$ where $m$ is an integer and $2m$ represents an even number.
$3n = 2m - 1$ this implies that $3n$ is an odd number, as $2m - 1$ represents an odd number.
Since an odd number multiplied by an even number gives an even number and an odd number multiplied by an odd number gives an odd number for the equation $3n$ can only equal an odd number if n is odd.

\section*{Question 3}\label{results}
In this section we describe the results.

\section{Conclusions}\label{conclusions}
We worked hard, and achieved very little.

\bibliographystyle{abbrv}
\bibliography{main}

\end{document}