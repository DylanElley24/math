\title{Assignment 1}

\date{\today}

\documentclass[12pt]{article}

\begin{document}
\maketitle


\section*{Question 1}
\begin{displaymath}
\begin{array}{|c c|c|c|c|}

P & Q & (P \land Q) & (P \lor Q) & (P \land Q) \Rightarrow (P \lor Q)\\
\hline 
T & T & T & T & T\\
T & F & F & T & T\\
F & T & F & T & T\\
F & F & F & F & T\\
\end{array}
\end{displaymath}

\paragraph{Conclution}
This truth table demonstrates that no matter the states of $P$ and $Q$ the results will always be true for the logical expression:\\ $(P \land Q) \Rightarrow (P \lor Q)$.

\section*{Question 2}
A much longer \LaTeXe{} example was written by Gil~\cite{Gil:02}.

\section{Results}\label{results}
In this section we describe the results.

\section{Conclusions}\label{conclusions}
We worked hard, and achieved very little.

\bibliographystyle{abbrv}
\bibliography{main}

\end{document}